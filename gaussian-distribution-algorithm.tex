\documentclass{article}
\usepackage[utf8]{inputenc}
\usepackage{amssymb}
\usepackage{amsmath}
\usepackage{amsthm}
\usepackage{hyperref}
%
\title{A simple algorithm for the cumulative distribution function of
the Gaussian distribution}
\author{Frank Recker\thanks{Frank Recker, General Reinsurance AG,
Theodor-Heuss-Ring 11, 50668 Köln, Germany, e-mail: {\tt
frank.recker@arcor.de}}}
\date{\today}
%
\newtheorem{algorithm}{Algorithm}
\newtheorem{lemma}[algorithm]{Lemma}
\newtheorem{theorem}[algorithm]{Theorem}
\newcommand\Rset{{\mathbb R}}
\newcommand\Nset{\mathbb N}
\newcommand\intd{{\mbox{d}}}
%
\begin{document}
\maketitle
\begin{abstract}
\noindent We give a simple algorithm for the value of
the cumulative distribution function of the Gaussian distribution.

\noindent {\bf Key Words:} Gaussian distribution, cumulative distribution
function, CDF

\noindent {\bf MSC2010 classification:} 60-04, 68-04
\end{abstract}
\tableofcontents

\section{Introduction}
The Gaussian Integral
\begin{equation}\Phi(x)=\frac{1}{\sqrt{2\pi}}\int\limits_{-\infty}^x\mbox{e}^{-\frac{t^2}{2}}\,\intd t\label{gausformel}\end{equation}
is ubiquitous in statistical applications. There are libraries such as
the GNU scientific library \cite{gnuscientific} which offer this
calculation as a function. However, sometimes one cannot use these
libraries due to technical or legal constraints. In this paper, we
describe a simple algorithm for this calculation. Example
implementations can be found at the github site \cite{github} of the
author.

\section{The Algorithm}
The algorithm for computing $\phi(x,n)$ for $x\in{\mathbb R}$ and
$n\in\Nset=\{1,2,\dots\}$ is defined as follows:
\begin{algorithm}\label{mainalgo}
  Let $d_n=0$ and
  \begin{equation*}d_j=-\frac{x^2}{2j}\left(d_{j+1}
     +\frac{1}{2j+1}\right)\end{equation*}
  for all $j=n-1,\dots,1$. Finally let
  \begin{equation*}\phi(x,n)=\frac{1}{2}
     +\frac{x}{\sqrt{2\pi}}\left(d_1+1\right).\end{equation*}
\end{algorithm}

The algorithm computes an approximation of the Gaussian cumulative
distribution function as defined in
Equation~(\ref{gausformel}). Theorem~\ref{algotheorem} formalizes
this.

\begin{theorem}\label{algotheorem}
Let $x\in{\mathbb R}$. Define $\Phi(x)$ as in
Equation~(\ref{gausformel}) and $\phi(x,n)$ for all $n\in\Nset$ as in
Algorithm~\ref{mainalgo}. Then we have
\begin{equation*}\lim_{n\to\infty}\phi(x,n)=\Phi(x).\end{equation*}
Furthermore if $n\ge\frac{x^2}{2}$ then
\begin{equation*}\left|\phi(x,n)-\Phi(x)\right|<\frac{1}{\sqrt{2\pi}}\frac{|x|^{2n+1}}{(2n+1)2^nn!}.\end{equation*}
\end{theorem}

Table~\ref{somevalues} shows the values of $\phi(x,n)$ and the maximum
error due to Theorem~\ref{algotheorem} for some~$x$ and~$n$. The
values were calculated with the code taken from~\cite{github}. Note that
all calculations are done with floating point numbers and hence
rounding errors will occur.

\begin{table}[ht]
\begin{center}
\begin{tabular}{*{3}{l|}l}
$x$ & $n$ & $\phi(x,n)$ & error bound\\
\hline
$1.96$ & 1 & 1.2819268695868082 & no bound\\
$1.96$ & 2 & 0.7812851592193613 & 0.28848977918213764\\
$1.96$ & 10 & 0.9749960638553972 & 7.014638266104427e-6\\
$1.96$ & 200 & 0.9750021048517796 & 1.23e-321\\
$5$ & 1 & 2.4947114020071637 & no bound\\
$5$ & 10 & -1169.2649270406318 & no bound\\
$5$ & 30 & 0.9285538915764981 & 9.958422559186228e-2\\
$5$ & 50 & 0.9999997133453642 & 4.5497179496632544e-12\\
$5$ & 200 & 0.9999997133486902 & 1.5200212487901728e-158
\end{tabular}
\end{center}
\caption{Some values and error bounds for $\phi(x,n)$\label{somevalues}}
\end{table}

\section{The Proof}\label{secproof}
First, we derive a power series for $\Phi(x)$.

\begin{lemma}
Let $\Phi(x)$ as in Equation~(\ref{gausformel}). Then
\begin{equation}\Phi(x)=\frac{1}{2}+\frac{1}{\sqrt{2\pi}}\sum_{k=0}^{\infty}\frac{(-1)^kx^{2k+1}}{(2k+1)2^kk!}.\label{gausformelpp}\end{equation}
\end{lemma}

\begin{proof}
We have $\Phi(0)=\frac{1}{2}$ and hence we can rewrite
Equation~(\ref{gausformel}) as
\begin{equation}\Phi(x)=\frac{1}{2}+\frac{1}{\sqrt{2\pi}}\int\limits_{0}^x\mbox{e}^{-\frac{t^2}{2}}\,\intd t.\label{gausformelp}\end{equation}
Next we use the power series for the $\exp$-function:
\begin{eqnarray*}\mbox{e}^x=\sum_{k=0}^{\infty}\frac{x^k}{k!}.\end{eqnarray*}
This series converges absolutely for all $x\in\Rset$. We insert this
into Equation~(\ref{gausformelp}) and get
\begin{equation*}\Phi(x)%
  =\frac{1}{2}+\frac{1}{\sqrt{2\pi}}\int\limits_{0}^x\sum_{k=0}^{\infty}\frac{\left(-\frac{t^2}{2}\right)^k}{k!}\,\intd t%
  =\frac{1}{2}+\frac{1}{\sqrt{2\pi}}\int\limits_{0}^x\sum_{k=0}^{\infty}\frac{(-1)^kt^{2k}}{2^kk!}\,\intd t%
.\end{equation*}
For convergent power series we can swap
integration and summation. Hence we get
\begin{equation*}\Phi(x)%
  =\frac{1}{2}+\frac{1}{\sqrt{2\pi}}\sum_{k=0}^{\infty}\int\limits_{0}^x\frac{(-1)^kt^{2k}}{2^kk!}\,\intd t%
  =\frac{1}{2}+\frac{1}{\sqrt{2\pi}}\sum_{k=0}^{\infty}\left[\frac{(-1)^kt^{2k+1}}{(2k+1)2^kk!}\right]_{t=0}^{t=x}%
\end{equation*}
and therefore
\[\Phi(x)=\frac{1}{2}+\frac{1}{\sqrt{2\pi}}\sum_{k=0}^{\infty}\frac{(-1)^kx^{2k+1}}{(2k+1)2^kk!}\]
as stated in the lemma.
\end{proof}

Next, we will show that $\phi(x,n)$ is in fact just a partial sum of
the series given in Equation~(\ref{gausformelpp}).

\begin{lemma}
Let $x\in\Rset$, $n\in\Nset$ and $\phi(x,n)$ as defined in
Algorithm~\ref{mainalgo}. Then we have
\begin{equation}\phi(x,n)=\frac{1}{2}+\frac{1}{\sqrt{2\pi}}\sum_{k=0}^{n-1}\frac{(-1)^kx^{2k+1}}{(2k+1)2^kk!}.\label{phigleichung}\end{equation}
\end{lemma}

\begin{proof}
For brevity define
\[\phi'(x,n)=\frac{1}{2}+\frac{1}{\sqrt{2\pi}}\sum_{k=0}^{n-1}\frac{(-1)^kx^{2k+1}}{(2k+1)2^kk!}.\]
For all $k\in\Nset$ let $a_k=\frac{1}{2k+1}$ and
$b_k=-\frac{x^2}{2k}$. Furthermore let $a_0=1$ and
$b_0=\frac{x}{\sqrt{2\pi}}$. Its easy to show by complete induction that
\[\phi'(x,n)=\frac{1}{2}+\sum_{k=0}^{n-1}\left(a_k\prod_{j=0}^kb_j\right).\]
Writing this in a Horner-scheme style gives
\[\phi'(x,n)=\frac{1}{2}+b_0\left(a_0+b_1\left(a_1+b_2\left(\dots+b_{n-2}\left(a_{n-2}+b_{n-1}a_{n-1}\right)\dots\right)\right)\right).\]
Algorithm~\ref{mainalgo} just evaluates this expression from the inner
brackets to the outer one ($d_j$ is the value of the term
$b_j(a_j+b_{j+1}(\dots))$ for all $j=1,\dots,n-2$).
\end{proof}

Now we can prove Theorem~\ref{algotheorem}.

\begin{proof}[Proof of Theorem~\ref{algotheorem}]
From Equations~(\ref{gausformelpp}) and (\ref{phigleichung}) we get
\[\lim_{n\to\infty}\phi(x,n)=\Phi(x).\]
The series $\phi(x,n)$ is alternating. Let $n\ge\frac{x^2}{2}$. We
compare the terms of the series for $k=n$ and $k=n-1$:
\[\left|\frac{\frac{(-1)^nx^{2n+1}}{(2n+1)2^nn!}}{\frac{(-1)^{n-1}x^{2n-1}}{(2n-1)2^{n-1}(n-1)!}}\right|%
=\left|\frac{(2n-1)x^2}{(2n+1)2n}\right|\le\left|\frac{2n-1}{2n+1}\right|<1.\]
Hence, the series for $\phi(x,n)$ is alternating and the absolute
values of the terms are monotone decreasing. Calculus shows
in this case that $|\phi(x,n)-\Phi(x)|$ is bound by the absolute value
of the $n$-th term of the series which is
\[\frac{1}{\sqrt{2\pi}}\frac{|x|^{2n+1}}{(2n+1)2^nn!}.\]
\end{proof}

\begin{thebibliography}{9}
\bibitem{gnuscientific}The Gaussian Distribution in the GNU
Scientific Library,
\url{http://www.gnu.org/software/gsl/manual/html\_node/The-Gaussian-Distribution.html}
\bibitem{github}Implementations of Algorithm~\ref{mainalgo} on github,
\url{https://github.com/frecker/gaussian-distribution}
\end{thebibliography}
\end{document}
